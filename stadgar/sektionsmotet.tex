\section{Sektionsmötet}
\subsection{Befogenheter}
Sektionsmötet är teknologsektionens högsta beslutande organ.
\subsection{Sammanträden}
Det skall hållas fyra ordinarie sektionsmöten, ett per läsperiod. Utöver detta kan extra sektionsmöten hållas.
\subsection{Utlysande}
\subsubsection{}
Sektionsmötet sammanträder på kallelse av sektionsstyrelsen.
\subsubsection{}
Rätt att hos sektionsstyrelsen begära utlysande av sektionsmöte tillkommer ledamot i sektionsstyrelsen, inspektor, kårens inspektor, kårstyrelsen, teknologsektionens revisorer eller minst 25 av teknologsektionens medlemmar. Sådant möte ska hållas inom tio läsdagar.
\subsubsection{}
\label{sec:sektionsmote_utlysande}
Sektionsmöte skall utlysas minst fem läsdagar i förväg genom att kallelse enligt reglemente anslås. Inkomna motioner och propositioner skall anslås minst tre läsdagar i förväg.
\subsection{Åligganden}
\subsubsection{}
Senast dagen före ordinarie mandatperiods början skall följande behandlas på sektionsmöte:
\begin{itemize}
\item Omfördelning av sektionens och föreningarnas tillgångar.
\item Val av sektionsstyrelse.
\item Val av revisorer.
\item Val av inspektor om så är aktuellt.
\end{itemize}
\subsubsection{}
Senast dagen före verksamhetsårets början skall följande behandlas på sektionsmöte:
\begin{itemize}
\item Sektionsavgift för de två kommande terminerna.
\item Fastställande av preliminär budget för nästkommande verksamhetsår.
\end{itemize}
\subsubsection{}
Senast sex månader efter verksamhetsårets början skall följande behandlas på sektionsmöte:
\begin{itemize}
\item Sektionens och sektionsföreningarnas års- och revisionsberättelse för föregående verksamhetsår.
\item Beslut om ansvarsfrihet.
\item Fastställande av budget för innevarande verksamhetsår.
\end{itemize}
\subsection{Beslutförhet}
\subsubsection{}
Sektionsmötet är beslutsmässigt om mötet är behörigt utlyst enligt stadgans kapitel~\ref{sec:sektionsmote_utlysande}.
\subsubsection{}
Om färre än 40 medlemmar är närvarande då beslut ska fattas, kan detta endast ske om ingen yrkar på bordläggning. Detsamma gäller beslut i frågor som ej har varit anslagna tre läsdagar i förväg.
\subsection{Motion}
Medlem som önskar ta upp fråga på föredragningslistan skall anmäla detta skriftligen till sektionsstyrelsen senast fem läsdagar före sektionsmöte.
\subsection{Överklagande}
Beslut av sektionsmötet som strider mot kårens eller sektionens stadga, reglemente, ekonomiska reglemente eller policy får undanröjas av kårfullmäktige. Sådant beslut ska tas upp till prövning om det begärs av en kårmedlem då det rör kårens stadga, eller sektionsmedlem då det rör teknologsektionens stadga, reglemente, ekonomiska reglemente eller policy.
\subsection{Omröstning}
\subsubsection{}
Röstning med fullmakt får ej ske.
\subsubsection{}
Omröstning skall ske öppet, om ej sluten votering begärs.
\subsubsection{}
Vid lika röstutfall äger mötesordförande utslagsröst, utom vid personval då lotten avgör.
\subsubsection{}
Då flera förslag ställs mot varandra skall röstningsförfarandet fastslås innan omröstning påbörjas.
\subsubsection{}
Alla frågor som behandlas på sektionsmötet avgörs med enkel röstövervikt om inget annat anges i stadgan. Nedlagda röster räknas ej.
\subsection{Närvaro- och yttranderätt}
Närvaro- och yttranderätt tillkommer medlem, hedersmedlem, kårstyrelseledamöter, inspektor, revisorer samt av mötet adjungerade icke-medlemmar.
\subsection{Förslagsrätt}
Förslagsrätt tillkommer medlem, inspektor samt av mötet adjungerade icke-medlemmar.
\subsection{Rösträtt}
Rösträtt tillkommer medlem.
\subsection{Protokoll}
Sektionsmötesprotokoll skall justeras av två av mötet valda justeringsmän. Justerat protokoll ska anslås senast tio läsdagar efter mötet.
